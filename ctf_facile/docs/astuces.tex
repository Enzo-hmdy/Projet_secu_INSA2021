\documentclass{article}
\usepackage[
backend=biber,
style=alphabetic,
sorting=ynt
]{biblatex}
\usepackage{geometry}
 \geometry{
 a4paper,
 total={170mm,257mm},
 left=20mm,
 top=20mm,
 }
\usepackage[french]{babel}
\usepackage{optidef}
\usepackage[T1]{fontenc}

\title{\textbf{CTF FACILE}\\ \textbf{Astuces}}
\author{Groupe 4}
\date{5 Novembre 2021}


\begin{document}
\maketitle

\section*{Ports ouverts:}

La première chose à faire est toujours de savoir quels ports sont ouverts sur la machine. Il faut donc utiliser une commande ou un outil permettant de les analyser.

\section*{Page login:}

Le bruteforce ne fonctionne pas ? Il peut être judicieux de se renseigner sur le top 10 de l'OWASP.\newline

Attention au service utilisé pour la base de donnée. Les caractères de commentaires peuvent être différents selon l'utilitaire.

\section*{Hashage:}

Les mots de passes de la base de données sont sûrement hachés. Il existe pléthore d'outils permettant de les dés-hachés.

\section*{SSH:}

Un port ssh d'ouvert, peut-être que nous pouvons nous connecter ? Ce serait bête que l'administrateur du serveur n'est pas utilisé un mot de passe différent que celui d'administration du site.

\end{document}